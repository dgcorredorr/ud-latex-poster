%%%%%%%%%%%%%%%%%%%%%%%%%%%%%%%%%%%%%%%%%%%%%%%%%%%%%%%%%%%%%%%%%%%%%%%%%%%%% 
%
% This is a LaTeX file for an A3 poster.
%
%%%%%%%%%%%%%%%%%%%%%%%%%%%%%%%%%%%%%%%%%%%%%%%%%%%%%%%%%%%%%%%%%%%%%%%%%%%%% 

%%%%%%%%%%%%%%%%%%%%%%%%%%%%%%%%%%%%%%%%%%%%%%%%%%%%%%%%%%%%%%%%%%%%%%%%%%%%% 
%%%%%%%%%%%%%%%%%%%%%%%%%%%%%%%%%%%%%%%%%%%%%%%%%%%%%%%%%%%%%%%%%%%%%%%%%%%%%
%
% Asymptotic behavior of solutions to the generalized Becker-Döring
% equations for general initial data.
%
% Poster for the HYKE-3 meeting in Rome, 13-15 April 2005.
% 
%%%%%%%%%%%%%%%%%%%%%%%%%%%%%%%%%%%%%%%%%%%%%%%%%%%%%%%%%%%%%%%%%%%%%%%%%%%%%
%%%%%%%%%%%%%%%%%%%%%%%%%%%%%%%%%%%%%%%%%%%%%%%%%%%%%%%%%%%%%%%%%%%%%%%%%%%%%

\documentclass{article}
\input{source/preambulo}
\usepackage{sectsty}
\sectionfont{\fontsize{60}{72}\selectfont}

% ===========================================================================

\title{Título con Iniciales en Mayúsculas Excepto Preposiciones y Sustantivos}
\author{}
\date{}

\begin{document}
%\maketitle

\begin{center}
  \begin{minipage}{.16\linewidth}
  \qquad\qquad
    \includegraphics[width=.7\linewidth]{media/images/logo_circular_negro.png}
    ~\vfill
    ~\vfill
  \end{minipage}
  %&
  \begin{minipage}{.6\linewidth}
    \begin{center}
       \textsf{\textbf{{\fontsize{90}{108}\selectfont Título con Iniciales en Mayúsculas Excepto Preposiciones y Sustantivos}}}\\\vspace{1cm}
       \textrm{\fontsize{40}{48}\selectfont Autor 1\Mark{1}, Autor 2\Mark{2}, Autor 3\Mark{3} y Autor 4\Mark{4} \\\vspace{5mm} \textit{Ingeniería Electrónica}, \textit{Universidad Distrital Francisco José de Caldas}\\\vspace{5mm} Email: \Mark{1}correoautor1@correo.udistrital.edu.co, \Mark{2}correoautor2@correo.udistrital.edu.co, \Mark{3}correoautor3@correo.udistrital.edu.co, \Mark{4}correoautor4@correo.udistrital.edu.co}  
    \end{center}
  \end{minipage}
  %&
  \hspace{.03\linewidth}
    \begin{minipage}{.16\linewidth}
  \qquad\qquad
    \includegraphics[width=.7\linewidth]{media/images/LOGO_IE.png}
    ~\vfill
    ~\vfill
  \end{minipage}
\end{center}

\vspace{2cm}

% ---------------------------------------------------------------------------
\fontsize{30}{36}\selectfont
\setlength{\columnsep}{2cm}
\begin{multicols}{3}
% ---------------------------------------------------------------------------
% En caso de no poner resumen se comenta la siguiente sección
% ---------------------------------------------------------------------------
\noindent
\fcolorbox{black}{azulcielo}{
  \begin{minipage}[H]{.96\linewidth}
    \begin{center}
      \vspace{1cm}
      \section*{Resumen}
        En esta sección se debe escribir un resumen de lo que contiene el documento en general. Debe describir el tópico, el alcance, los hallazgos principales y las conclusiones más relevantes. De hecho, esta sección debe ser la última parte que se escribe. No debe contener abreviaturas, siglas ni fórmulas sino texto plano sencillo.  Su propósito fundamental es comunicar al lector la esencia del reporte y tener una idea del contenido del documento.
      \vspace{1cm}
    \end{center}
  \end{minipage}
  
}
\section*{Introducción}

\noindent La introducción es la sección donde se describe el contexto del problema y por qué es de interés. Esta sección debe describir clara pero brevemente información sobre el problema, qué se ha hecho antes y los objetivos del proyecto actual. Es una sección que contiene el mayor número de citas puesto que recurre a documentos ya publicados.

La introducción está compuesta de cinco elementos: (a) apertura del objeto en estudio, (b) revisión del trabajo previo relevante para el desarrollo del proyecto descrito en este reporte, (c) descripción del problema tratado en este reporte y cómo se relaciona con el trabajo previo, (d) descripción de las suposiciones y restricciones establecidas para el desarrollo del proyecto y (e) perspectiva general del reporte mencionando el contenido de cada sección. La introducción no debe abarcar más del 25\% del total del reporte.

\section*{Formulación del problema \\ y propuesta de solución}

\noindent En esta sección es donde se hace la descripción formal y detallada del problema que se va a resolver y, en consecuencia, la propuesta de solución. Define toda la terminología y la notación utilizada en el reporte. Sin embargo, es común que la terminología y la notación se definan junto con la descripción del problema. Si es necesario, se deben describir también los sistemas experimentales implementados.

Esta sección debería incluir suficiente análisis teórico o matemático que permita derivaciones y resultados numéricos para ser evaluados. Debería también incluir descripción de procedimientos, técnicas, instrumentación, precauciones especiales y otros. Si la parte experimental es larga y detallada, esta puede ser colocada al final del reporte, como un apéndice, de manera que no interrumpa el flujo conceptual de la sección. Su ubicación dependerá de la naturaleza del proyecto y de la discreción del escritor. Sin embargo, cualquiera que sea su ubicación, debe hacerse referencia a ella dentro del documento.

% no es posible utilizar el entorno figure dentro del entorno multicol, por lo que se ubican manualmente las imágenes al inicio o al final de cada columna 
\begin{figure}[H]
    \vspace{5mm}
    \centering
    \includegraphics[width=0.9\linewidth]{example-image-a}
    \caption{Imágen de prueba A}
\end{figure}
% ---------------------------------------------------------------------------
\section*{Resultados}

\noindent En esta sección se muestra la solución o soluciones planteadas del problema junto con los resultados obtenidos. Se sintetizan los datos relevantes, las observaciones y los hallazgos. También es la sección donde se desarrollan todas las ecuaciones sobre las cuales estarán basados los resultados. Es una sección que debe estar suficientemente detallada para que otros investigadores puedan repetir el trabajo y obtener resultados comparables.

Para presentar resultados claros y concisos, se pueden usar tablas de datos, gráficas, ecuaciones, diagramas y figuras. También pueden ser utilizados esquemas que muestren secuencias de reacción. Estos se refieren a diagramas que muestren tendencias de los resultados cuando se están haciendo pruebas para encontrar o definir parámetros apropiados. Todos los objetos que se utilicen para ilustrar los resultados deben ser referidos desde el texto.
% no es posible utilizar el entorno figure dentro del entorno multicol, por lo que se ubican manualmente las imágenes al inicio o al final de cada columna 
\begin{figure}[H]
    \vspace{5mm}
    \centering
    \includegraphics[width=0.9\linewidth]{example-image-b}
    \caption{Imágen de prueba B}
\end{figure}

\section*{Discusión}

\noindent La esencia del reporte es el análisis y la interpretación de los resultados. ¿Qué significan? ¿Cómo se relacionan con los objetivos del proyecto? ¿Hasta qué punto ellos (los resultados) han resuelto el problema? Detalles como estos son los que hacen parte de esta sección.
% ---------------------------------------------------------------------------
%


\renewcommand{\bibfont}{\fontsize{30}{36}\selectfont}
\nocite{*}
\printbibliography



\end{multicols}

\end{document}
